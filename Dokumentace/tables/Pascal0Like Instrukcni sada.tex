\documentclass[
12pt,
a4paper,
pdftex,
czech
]{report}

\usepackage{float}
\usepackage[czech]{babel}
\usepackage[utf8]{inputenc}
\usepackage{lmodern}
\usepackage{textcomp}
\usepackage[T1]{fontenc}
\usepackage{amsfonts}
\usepackage{titlesec}
\usepackage{graphicx}
\usepackage{longtable}
\usepackage{multirow}
\usepackage{tabularx}
\usepackage[unicode]{hyperref}


\begin{document}

\begin{longtable}{|c|l|p{10cm}|}
\caption{Instrukce}
\label{instrukce}
\endfirsthead
\endhead
\hline
		Kód & Instrukce & Popis \\
\hline\hline
\rule{0pt}{3ex}1 & \texttt{LIT 0, M} & Vloží konstantní celou hodnotu (\texttt{literál}) \texttt{M} do zásobníku \\ \hline
\rule{0pt}{3ex}2 & \texttt{LRT 0, M} & Vloží konstantní reálnou hodnotu (\texttt{literál}) \texttt{M} do zásobníku \\ \hline
\rule{0pt}{3ex}3 & \texttt{OPR 0, M} & \textbf{Operace}, která se provede nad vrcholem zásobníku \textbf{pro celé čísla} \\ \hline
\rule{0pt}{3ex} & \texttt{OPR 0, 1} & \textbf{Negace}; vybere vrchol a vrátí negativní hodnotu \\ \hline
\rule{0pt}{3ex} & \texttt{OPR 0, 2} & \textbf{Sčítání}; vybere dvě hodnoty, sečte a vrátí \\ \hline
\rule{0pt}{3ex} & \texttt{OPR 0, 3} & \textbf{Odečítání}; vybere dvě hodnoty, odečte druhou první a vrátí výsledek \\ \hline
\rule{0pt}{3ex} & \texttt{OPR 0, 4} & \textbf{Násobení}; vybere dvě hodnoty, vynásobí a vrátí výsledek \\ \hline
\rule{0pt}{3ex} & \texttt{OPR 0, 5} & \textbf{Dělení}; vybere dvě hodnoty, vydělí druhou první \\ \hline
\rule{0pt}{3ex} & \texttt{OPR 0, 6} & \textbf{Lichost}; vybere vrchol a vloží 1 když liché, 0 když sudé \\ \hline
\rule{0pt}{3ex} & \texttt{OPR 0, 7} & \textbf{Modulo}; vybere dvě hodnoty, vydělí druhý prvním a vloží zbytek \\ \hline
\rule{0pt}{3ex} & \texttt{OPR 0, 8} & \textbf{Rovnost}; vybere dvě hodnoty, a vloží 1 pokud se rovnají, jinak 0 \\ \hline
\rule{0pt}{3ex} & \texttt{OPR 0, 9} & \textbf{Nerovnost}; vybere dvě hodnoty a vloží 0 pokud se rovnají, jinak 0 \\ \hline
\rule{0pt}{3ex} & \texttt{OPR 0, 10} & \textbf{Menší než}; vybere dvě hodnoty a vloží 1 pokud je první menší než druhá, jinak 0 \\ \hline
\rule{0pt}{3ex} & \texttt{OPR 0, 11} & \textbf{Větší nebo rovno než}; vybere dvě hodnoty a vloží 1 pokud je první větší nebo rovno než druhá, jinak 0 \\ \hline
\rule{0pt}{3ex} & \texttt{OPR 0, 12} & \textbf{Větší}; vybere dvě hodnoty a vloží 1 pokud je první větší nebo rovno než druhá, jinak 0 \\ \hline
\rule{0pt}{3ex} & \texttt{OPR 0, 13} & \textbf{Menší nebo rovno než}; vybere dvě hodnoty a vloží 1 pokud je první menší nebo rovno než druhá, jinak 0 \\ \hline
\rule{0pt}{3ex}4 & \texttt{LOD L, M} & \textbf{Načtení}; načte hodnotu vrcholu z umístění dané offsetem M od L lexikografických úrovní dolů \\ \hline
\rule{0pt}{3ex}5 & \texttt{STO L, M} & \textbf{Uložení}; uloží hodnotu vrcholu z umístění dané offsetem M od L lexikografických úrovní dolů \\ \hline
\rule{0pt}{3ex}6 & \texttt{CAL L, M} & \textbf{Volání procedury} v kódovém indexu \textbf{M} \\ \hline
\rule{0pt}{3ex}7 & \texttt{RET 0, 0} & \textbf{Návrat z procedury}; vrátí se z procedury do volající procedury \\ \hline
\rule{0pt}{3ex}8 & \texttt{INT 0, M} & \textbf{Alokování} místa pro M hodnot na vrcholo zásobníku \\ \hline
\rule{0pt}{3ex}9 & \texttt{JMP 0, M} & Provede skok do instrukce \textbf{M} \\ \hline
\rule{0pt}{3ex}10 & \texttt{JMC 0, M} & Vybere vrchol a skočí k instrukci M pokud je rovna 0, \textbf{podmíněný skok} \\ \hline
\rule{0pt}{3ex}11 & \texttt{REA L, M} & \textbf{Načte celé číslo} ze vstupu a uloží jej na zásobník \\ \hline
\rule{0pt}{3ex}12 & \texttt{WRI L, M} & Odebere celé číslo z vrcholu zásobníku a \textbf{vypíše jej na vstup} \\ \hline
\rule{0pt}{3ex}13 & \texttt{RER L, M} & \textbf{Načte} reálné číslo ze vstupu a uloží jej na zásobník \\ \hline
\rule{0pt}{3ex}14 & \texttt{WRR L, M} & Odebere reálné číslo z vrcholu zásobníku a \textbf{vypíše jej na vstup} \\ \hline
\rule{0pt}{3ex}15 & \texttt{OPF 0, M} & \textbf{Operace}, která se provede nad vrcholem zásobníku \textbf{pro reálná čísla} \\ \hline
\rule{0pt}{3ex} & \texttt{OPF 0, 1} & \textbf{Negace}; vybere vrchol a vrátí negativní hodnotu \\ \hline
\rule{0pt}{3ex} & \texttt{OPF 0, 2} & \textbf{Sčítání}; vybere dvě hodnoty, sečte a vrátí \\ \hline
\rule{0pt}{3ex} & \texttt{OPF 0, 3} & \textbf{Odečítání}; vybere dvě hodnoty, odečte druhou první a vrátí výsledek \\ \hline
\rule{0pt}{3ex} & \texttt{OPF 0, 4} & \textbf{Násobení}; vybere dvě hodnoty, vynásobí a vrátí výsledek \\ \hline
\rule{0pt}{3ex} & \texttt{OPF 0, 5} & \textbf{Dělení}; vybere dvě hodnoty, vydělí druhou první \\ \hline
\rule{0pt}{3ex} & \texttt{OPF 0, 6} & \textbf{Lichost}; vybere vrchol a vloží 1 když liché, 0 když sudé \\ \hline
\rule{0pt}{3ex} & \texttt{OPF 0, 7} & \textbf{Modulo}; vybere dvě hodnoty, vydělí druhý prvním a vloží zbytek \\ \hline
\rule{0pt}{3ex} & \texttt{OPF 0, 8} & \textbf{Rovnost}; vybere dvě hodnoty, a vloží 1 pokud se rovnají, jinak 0 \\ \hline
\rule{0pt}{3ex} & \texttt{OPF 0, 9} & \textbf{Nerovnost}; vybere dvě hodnoty a vloží 0 pokud se rovnají, jinak 0 \\ \hline
\rule{0pt}{3ex} & \texttt{OPF 0, 10} & \textbf{Menší než}; vybere dvě hodnoty a vloží 1 pokud je první menší než druhá, jinak 0 \\ \hline
\rule{0pt}{3ex} & \texttt{OPF 0, 11} & \textbf{Větší nebo rovno než}; vybere dvě hodnoty a vloží 1 pokud je první větší nebo rovno než druhá, jinak 0 \\ \hline
\rule{0pt}{3ex} & \texttt{OPF 0, 12} & \textbf{Větší}; vybere dvě hodnoty a vloží 1 pokud je první větší nebo rovno než druhá, jinak 0 \\ \hline
\rule{0pt}{3ex} & \texttt{OPF 0, 13} & \textbf{Menší nebo rovno než}; vybere dvě hodnoty a vloží 1 pokud je první menší nebo rovno než druhá, jinak 0 \\ \hline
\rule{0pt}{3ex} 16 & \texttt{RTI 0, 0} & \textbf{Reálné číslo na celé číslo}; vybere jednu hodnotu ze zásobníku a vloží celou část čísla do zásobníku \\ \hline
\rule{0pt}{3ex} 17 & \texttt{ITR 0, 0} & \textbf{Celé číslo na reálné číslo}; vybere jednu hodnotu ze zásobníku a vloží číslo jako reálné do zásobníku \\ \hline
\rule{0pt}{3ex} 18 & \texttt{NEW 0, 0} &  \textbf{Alokace na haldě}; alokuje se jedno místo na haldě, na zásobník vloží hodnotu představující pozici místa v haldě \\ \hline
\rule{0pt}{3ex} 19 & \texttt{DEL 0, 0} & \textbf{Uvolnění místa na haldě}; odebere ze zásobníku jednu hodnotu a to adresu na haldě, kterou uvolní \\ \hline
\rule{0pt}{3ex} 20 & \texttt{LDA 0, 0} & \textbf{Načtení hodnoty z haldy}; odebere ze zásobníku hodnotu a vloží hodnotu z haldy \\ \hline
\rule{0pt}{3ex} 21 & \texttt{STA 0, 0} & \textbf{Uložení hodnoty na haldu}; odebere dvě hodnoty zásobníku. Na první představující adresu uloží druhou v haldě \\ \hline
\rule{0pt}{3ex} 22 & \texttt{PLD 0, 0} & \textbf{Dynamické načtení hodnoty z místa určeného L/A}; odebere ze zásobníku dvě hodnoty. První je úroveň zanoření a druhá je relativní pozice \\ \hline
\rule{0pt}{3ex} 23 & \texttt{PST 0, 0} & \textbf{Dynamické uložení hodnoty z místa určeného L/A}; odebere ze zásobníku tři hodnoty. První je úroveň zanoření, druhá relativní pozice a třetí \\ \hline
\end{longtable}
\end{document}